\documentclass{beamer}

\usepackage[utf8]{inputenc}
\usepackage[T1]{fontenc}
\usepackage{graphicx}
\usepackage{hyperref}
\usepackage{listings}
\usepackage{color}

% Tema de Beamer
\usetheme{Madrid}
\usecolortheme{seahorse}

% Título
\title{Aprendiendo Redes Neuronales con ChatGPT}
%\subtitle{Explorando el perceptrón y más allá}
\author{Dorys Trujillo Beltrán}
\institute{Prompt Engineering}
\date{10 de noviembre de 2023}

% Configuración del pie de página
%\setbeamertemplate{footline}{\author{} \institute{} \date{\insertdate}}

%\setbeamertemplate{footline}{\author{}}
\setbeamertemplate{footline}{%
    \leavevmode%
    \hbox{%
        \begin{beamercolorbox}[wd=.33\paperwidth,ht=2.25ex,dp=1ex,center]{author in head/foot}%
            \usebeamerfont{author in head/foot}\insertauthor
        \end{beamercolorbox}%
        \begin{beamercolorbox}[wd=.33\paperwidth,ht=2.25ex,dp=1ex,center]{title in head/foot}%
            \usebeamerfont{title in head/foot}\insertinstitute
        \end{beamercolorbox}%
        \begin{beamercolorbox}[wd=.33\paperwidth,ht=2.25ex,dp=1ex,right]{date in head/foot}%
            \usebeamerfont{date in head/foot}\insertdate\hspace*{2ex}
        \end{beamercolorbox}%
    }%
    \vskip0pt%
}


\begin{document}

% Título de la diapositiva de título
\begin{frame}
\titlepage
\end{frame}

% Diapositiva de Contenido
\begin{frame}
\frametitle{Contenido}
\tableofcontents
\end{frame}

% Sección 1: Introducción
\section{Introducción}

% Diapositiva 1
\begin{frame}
\frametitle{Introducción}
\begin{itemize}
\item Las redes neuronales son una parte fundamental del aprendizaje profundo.
\item ChatGPT es una poderosa herramienta para comprender conceptos de redes neuronales.
\item En esta presentación, exploraremos el concepto de perceptrón y su implementación.
\end{itemize}
\end{frame}

% Sección 2: Perceptrón
\section{Perceptrón}

% Diapositiva 2
\begin{frame}
\frametitle{El Perceptrón}
\begin{itemize}
\item El perceptrón es una unidad básica en una red neuronal.
\item Funciona como un clasificador binario.
\item Echemos un vistazo a su código y cómo funciona.
\end{itemize}
\end{frame}

% Código del Perceptrón
\begin{frame}[fragile]
\frametitle{Código del Perceptrón}
\begin{lstlisting}[language=Python]
def perceptron(x1,x2):
    v = x1 + x2 - 1.2
    if v < 0:
        return 0
    else:
        return 1

X = [[0, 0],
    [0, 1],
    [1, 0],
    [1, 1]]

for n in range(4):
    y = perceptron(X[n][0],X[n][1])
    print('Input :', X[n], 'Output :', y)
\end{lstlisting}
\end{frame}

% Diapositiva 3
\begin{frame}
\frametitle{Explicación de ChatGPT}
\begin{itemize}
\item \textbf{def perceptron(x1, x2):} Esto define una función llamada perceptron que toma dos argumentos, x1 y x2.

\item \textbf{v = x1 + x2 - 1.2:}: En esta línea, se calcula un valor v sumando x1 y x2 y luego restando 1.2 a esa suma. Este valor v se usa en la siguiente línea para tomar una decisión.

\item \textbf{if v < 0:} Aquí se inicia una estructura condicional. Si el valor v es menor que 0, se ejecutará el bloque de código dentro de esta condición.

\item \textbf{return 0:} En caso de que v sea menor que 0, la función perceptron devuelve 0. Esto significa que si la suma de x1 y x2 menos 1.2 es menor que 0, la función retornará 0.

\item \textbf{else:}: Si v no es menor que 0, se ejecutará el bloque de código dentro de este bloque else.

\item \textbf{return 1:} En este caso, la función perceptron devuelve 1. Esto significa que si la suma de x1 y x2 menos 1.2 no es menor que 0, la función retornará 1.

\end{itemize}
\end{frame}


% Resultados del Perceptrón
\begin{frame}
\frametitle{Explicación de ChatGPT}
\begin{itemize}

\item \textbf{X = [[0, 0], [0, 1], [1, 0], [1, 1]]:} Esto crea una lista X que contiene cuatro sub-listas, cada una de las cuales tiene dos valores. Estas sub-listas representan las entradas del perceptrón.

\item \textbf{for n in range(4):}: Se inicia un bucle for que recorre los valores de n desde 0 hasta 3. Esto se hace para evaluar el perceptrón en cada una de las cuatro combinaciones de entrada.

\item \textbf{y = perceptron(X[n][0], X[n][1]):} En cada iteración del bucle, se llama a la función perceptron con los valores de X[n][0] y X[n][1] como argumentos. Esto calcula la salida del perceptrón para la entrada actual y la almacena en la variable y.

\item \textbf{print('Input :', X[n], 'Output :', y):} Se imprime en la consola un mensaje que muestra la entrada actual y la salida del perceptrón. Por ejemplo, para la primera entrada [0, 0], mostrará "Input: [0, 0] Output: 0" si la salida es 0 según la función perceptron. El proceso se repite para todas las combinaciones de entrada en la lista X.

\end{itemize}
\end{frame}

% Sección 3: Aprendizaje Profundo
%\section{Aprendizaje Profundo}

% Diapositiva 4
%\begin{frame}
%\frametitle{Aprendizaje Profundo}
%\begin{itemize}
%\item El perceptrón es solo el comienzo.
%\item Explora conceptos más avanzados de redes neuronales.
%\item ChatGPT es una herramienta valiosa para aprender y experimentar.
%\end{itemize}
%\end{frame}

% Diapositiva de Cierre
\begin{frame}
\frametitle{Gracias}
\begin{center}
\Huge ¡Gracias por su atención!
\end{center}
\end{frame}

\end{document}

